	\documentclass[11pt, oneside]{article}   	% use "amsart" instead of "article" for AMSLaTeX format
\usepackage{geometry}                		% See geometry.pdf to learn the layout options. There are lots.
\geometry{letterpaper}                   		% ... or a4paper or a5paper or ... 
%\geometry{landscape}                		% Activate for for rotated page geometry
%\usepackage[parfill]{parskip}    		% Activate to begin paragraphs with an empty line rather than an indent
\usepackage{graphicx}				% Use pdf, png, jpg, or eps� with pdflatex; use eps in DVI mode
								% TeX will automatically convert eps --> pdf in pdflatex
\usepackage{amsmath}								
\usepackage{color}
\usepackage{mathtools}
\usepackage{amssymb}

\usepackage{hyperref}

\hypersetup{
    bookmarks=true,%
    colorlinks,%
    citecolor=blue,%
    filecolor=black,%
    linkcolor=blue,%
    urlcolor=blue
}


\usepackage{overpic}
\usepackage{array}
\usepackage{rotating}


\begin{document}
\today
\\
\\
\section*{Instabilities in first order fluid theories}
In these notes we aim to expand on the work of \cite{Hiscock:1983zz}. In \cite{Hiscock:1983zz} the authors study instabilities in first order fluid theories. They construct the most general Lorentzian fluid energy momentum tensor one can construct up to first order, whilst fullfilling the positive entropy condition. It turns out that in the linearized spectrum, one can see the onset of instabilities. 

%%
%%
\subsection*{What is the relativistic fluid tensor with non-negative entropy and non-specified frame?}
%%
%%
The relativistic fluid with non-negative entropy and non-specified frame is given by \cite{Hiscock:1983zz}:
\begin{equation}\label{eq:emt1}
	T^{\mu\nu}
	=
	\mathcal{E}u^{\mu}u^{\nu}
	+
	P\Delta^{\mu\nu}
	-
	\zeta \partial_{\gamma}u^{\gamma}\Delta^{\mu\nu}
	+
	q^{\mu}u^{\nu}
	+
	q^{\nu}u^{\mu}
	-
	\eta \sigma^{\mu\nu}
	\,,
\end{equation}
\begin{equation}\label{eq:emt2}
	J^{\mu}
	=
	nu^{\mu}
	-
	\sigma T \Delta^{\mu\alpha}\partial_{\alpha}\left(\frac{\mu}{T}\right)
	\,,
\end{equation}	
where $\sigma$ is the usual traceless tensor and we have the relations $\mathcal{E}+P=sT+\mu n$, $\Delta^{\mu}_{\;\;\nu}=\delta^{\mu}_{\;\;\nu}+u^{\mu}u_{\nu}$, $u^{2}=-1$ and $q^{\mu}=-\kappa T \Delta^{\mu\alpha}\left[\frac{1}{T}\partial_{\alpha}T+u^{\gamma}\partial_{\gamma}u_{\alpha}\right]$. Here $\zeta$, $\eta$, $\kappa$, $\sigma\geq0$ and these coefficients are respectively named bulk viscosity, shear viscosity, thermal conductivity and particle diffusion. If one puts $\sigma=0$, the Eckart frame is obtained. The Landau frame is obtained by setting $\kappa=0$.

Could we have guessed this purely from considering the results for Landau and Eckart frame? We present the Eckart frame result \cite{Kovtun:2012rj}
\begin{equation}
	T^{\mu\nu}
	=
	\mathcal{E}u^{\mu}u^{\nu}
	+
	P\Delta^{\mu\nu}
	-
	\zeta \partial_{\gamma}u^{\gamma}\Delta^{\mu\nu}
	+
	q^{\mu}u^{\nu}
	+
	q^{\nu}u^{\mu}
	-
	\eta \sigma^{\mu\nu}
	\,,
\end{equation}
\begin{equation}
	J^{\mu}
	=
	nu^{\mu}
	\,,
\end{equation}	
and the Landau frame result \cite{Kovtun:2012rj}
\begin{equation}
	T^{\mu\nu}
	=
	\mathcal{E}u^{\mu}u^{\nu}
	+
	P\Delta^{\mu\nu}
	-
	\zeta \partial_{\gamma}u^{\gamma}\Delta^{\mu\nu}
	-
	\eta \sigma^{\mu\nu}
	\,,
\end{equation}
\begin{equation}
	J^{\mu}
	=
	nu^{\mu}
	-
	\sigma T \Delta^{\mu\alpha}\partial_{\alpha}\left(\frac{\mu}{T}\right)
	\,.
\end{equation}	
The conclusion is that, indeed, by combining these two tensors, one finds the most general result written in \eqref{eq:emt1} and \eqref{eq:emt2}.
%%
%%
\subsection*{Guessing frameless Galilean fluid tensor with nonnegative entropy}
%%
%%
From \cite{Jensen:2014ama} we obtain the following results. Eckart frame is given by
\begin{equation}
	\mathcal{E}^{0}
	\equiv
	T^{0}_{\;\;0}
	=
	\mathcal{E}
	+
	\frac{1}{2}\rho v^{2}
	\,,
	\quad
	\mathcal{E}^{i}
	\equiv
	T^{i}_{\;\;0}
	=
	\left(
		\mathcal{E}+P+\frac{1}{2}\rho v^{2}
	\right)v^{i}
	-
	\eta \sigma^{ij}v_{j}
	-
	\zeta \partial_{k}v^{k}v^{i}
	-
	\kappa
	\partial_{i}T
	\,,
\end{equation}
\begin{equation}
	\mathcal{P}_{0}
	=
	\rho
	\,,
	\quad
	\mathcal{P}_{i}
	\equiv
	T_{\;\;i}^{0}
	=
	\rho v_{i}
	\,,
\end{equation}
\begin{equation}
	T_{ij}
	=
	P\delta_{ij}
	+
	\rho v_{i}v_{j}
	-
	\eta \sigma_{ij}
	-
	\zeta \delta_{ij}\partial_{k}v^{k}
	\,,
\end{equation}
\begin{equation}
	J^{0}
	=
	\rho
	\,,
	\quad
	J^{i}
	=
	\rho
	v^{i}
	\,.
\end{equation}
This solution is confirmed by, e.g., equation (1.5) in \cite{Kovtun:2012rj}. Using the conservation equations gives Navier-Stokes.

We now present the Landau frame
\begin{equation}
	\mathcal{E}^{0}
	\equiv
	T^{0}_{\;\;0}
	=
	\mathcal{E}
	+
	\frac{1}{2}\rho v^{2}
	-
	\sigma T v_{i}\partial_{i}
	\left(
		\frac{\mu}{T}
	\right)
	\,,
\end{equation}
\begin{equation}
	\mathcal{E}^{i}
	\equiv
	T^{i}_{\;\;0}
	=
	\left(
		\mathcal{E}+P+\frac{1}{2}\rho v^{2}
	-
	\sigma T v_{j}\partial_{j}
	\left(
		\frac{\mu}{T}
	\right)
	\right)v^{i}
	-
	\eta \sigma^{ij}v_{j}
	-
	\zeta \partial_{k}v^{k}v^{i}
	-
	\kappa
	\partial_{i}T
		-
	\sigma T v^{2}\partial_{i}
	\left(
		\frac{\mu}{T}
	\right)
	\,,
\end{equation}
\begin{equation}
	\mathcal{P}_{0}
	=
	\rho
	\,,
	\quad
	\mathcal{P}_{i}
	\equiv
	T_{\;\;i}^{0}
	=
	\rho v_{i}
		-
	\sigma T \partial_{i}
	\left(
		\frac{\mu}{T}
	\right)
	\,,
\end{equation}
\begin{equation}
	T_{ij}
	=
	P\delta_{ij}
	+
	\rho v_{i}v_{j}
	-
	\eta \sigma_{ij}
	-
	\zeta \delta_{ij}\partial_{k}v^{k}
	-
	\sigma T 
	\left(
		v_{i}\partial_{j}
		\left(
		\frac{\mu}{T}
		\right)
		+
		v_{j}\partial_{i}
		\left(
		\frac{\mu}{T}
		\right)
	\right)
	\,,
\end{equation}
\begin{equation}
	J^{0}
	=
	\rho
	\,,
	\quad
	J^{i}
	=
	\rho
	v^{i}
	-
	\sigma T \partial_{i}
	\left(
		\frac{\mu}{T}
	\right)
	\,.
\end{equation}
%%
%%
\subsection*{Attention}
%%
%%
not all coefficients are independent, are they?

\bibliographystyle{JHEP-2}
\bibliography{ref}
\end{document}








